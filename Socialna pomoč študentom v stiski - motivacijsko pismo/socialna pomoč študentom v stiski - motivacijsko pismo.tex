\documentclass[a4paper]{scrlttr2}
\usepackage{lmodern}
\usepackage[utf8]{inputenc}
\usepackage[T1]{fontenc}
\usepackage[slovene]{babel}
\usepackage{url}
\usepackage{eurosym}

\enlargethispage{3\baselineskip}
\voffset = 20pt

 
 
\setkomavar{fromname}{Igor Klepić}
\setkomavar{fromaddress}{Ulica Slavka Gruma 92 \\ 8000 Novo mesto}
\setkomavar{fromphone}{0123 45679}

 
\begin{document}
 
\begin{letter}{Fundacija Študentski tolar \\ Ustanova ŠOU v Ljubljani \\ Kersnikova 4 \\ 1000 Ljubljana}
 
\KOMAoptions{fromphone=false,fromfax=false}
\setkomavar{subject}{Prošnja za dodelitev enkratne izredne denarne pomoči študentom v stiski.}

\opening{Spoštovani gospod ali gospa,}
sem absolvent dodiplomskega študija programa matematika in računalništvo na fakulteti za računalništvo in informatiko, ki ga fakulteta skupaj izvaja s fakulteto za matematiko in fiziko v Ljubljani. Študij mi je izredno všeč in junija nameravam diplomirati. Na Vas se obračam z upanjem, da mi boste lahko pomagali pri reševanju hude socialne stiske, ki je v tem trenutku nevzdržna.
\\
\\
Živim v skupnem gospodinjstvu z materjo, ki je osem let ločena in je prejemnik invalidske pokojnine za štiri urni delovni čas, v višini cca. 230\euro \hspace{1pt} mesečno. Druge štiri ure je zaposlena kot proizvodnja delavka v Adrii Mobil, d.o.o., Novo mesto. Zaradi številnih zdravstvenih omejitev dela na prilagojenem delovnem mestu, vendar je zaradi pogostih akutnih zdravstvenih težav v bolniškem staležu. Njena plača za štiri ure dela zato dosega povprečno 300\euro \hspace{1pt} mesečno. 
\\
\\
Zaradi materinih zdravstvenih težav, ki izhajajo iz akutnega obolenja hrbtenice, išjasa in psihičnih težav, je bila večkrat hospitalizirana in ne more sprejeti nobenih dodatnih del, ki bi lahko zadostile finančno stisko.
\\
\\
Po očetu prejemam preživnino v višini 192\euro, torej imava z materjo skupne mesečne prihodke v višini 722\euro. Ob poplačilu najemnine za stanovanje, najemnine za študentsko sobo in splošnih položnic ne ostane za preživetje skoraj nič. 
\\
\\
Zaradi intenzivnega študija in želje, da v tekočih rokih opravim vse študijske obveznosti po študijskem programu si ne morem privoščiti občasnih del. Rad bi čimprej končal študij, da razbremenim svojo mater.
\\
\\

\pagebreak
Od očeta ne morem pričakovati dodatnih finančnih sredstev, saj je po dvoletni hospitalizaciji psihiatrične klinike Ljubljana nastanjen v socialno varstvenem zavodu Hrastovec pri Lenartu. Zaradi bolezni je invalidsko upokojen in preostanek pokojnine po poplačilu preživnine ne zadošča niti za plačilo oskrbe v domu. 
\\
\\
Zaradi minimalnega preseganja cenzusa v višini nekaj \euro \hspace{1pt} sem izgubil državno štipedijo, ki sem jo prejemal do oktobra 2012. 
\\
\\
Trenutno me najbolj boli dejstvo, da si sposojam denar za plačilo študenstke sobe, za prevoz od Novega mesta do Ljubljane, za šolske in študijske pripomočke, med katere sodi tudi prenosni računalnik, ki se mi zaradi obremenjenosti neprenehoma kvari. Ob vsem tem si tudi ne morem privoščiti obiska svojega očeta, saj nimam denarja za pot, kar me psihično izredno utruja in sem pred nekaj dnevi zaradi tega obiskal psihiatrično ambulanto. Zaradi navedene stiske se čedalje manj družim z vrstniki, postajam zamorjen, zadržujem se v sobi in po glavi mi rojijo temačne misli. Ob vsem skupaj si na dan komaj privoščim en topel obrok preko študentskih bonov in si ne morem privoščiti vseh osnovnih življenjskih potrebščin. 
\\
\\
Na podlagi zgoraj navedenega, vljudno prosim za dodelitev enkratne izredne denarne pomoči študentom v stiski v višini 400\euro, za kar Vam bom izredno hvaležen. Predvsem pa mi bo ta pomoč pomagala, da bom poplačal svoje zgoraj navedene obveznosti, uspešno zaključil študij in se zaposlil, kar je pravzaprav trenutno moje osebno poslanstvo.

\closing{S spoštovanjem,}

\vspace{5mm}





 
\end{letter}
 
\end{document}