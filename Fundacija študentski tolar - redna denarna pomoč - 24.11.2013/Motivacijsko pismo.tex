\documentclass[a4paper]{scrlttr2}
\usepackage{lmodern}
\usepackage[utf8]{inputenc}
\usepackage[T1]{fontenc}
\usepackage[slovene]{babel}
\usepackage{url}
\usepackage{eurosym}

\enlargethispage{3\baselineskip}
\voffset = 20pt
\setlength{\parindent}{0cm}

 
 
\setkomavar{fromname}{Igor Klepić}
\setkomavar{fromaddress}{Ulica Slavka Gruma 92 \\ 8000 Novo mesto \\ Slovenija}
\setkomavar{fromphone}{0123 45679}

 
\begin{document}
 
\begin{letter}{Fundacija Študentski tolar \\ Ustanova ŠOU v Ljubljani \\ Kersnikova 4 \\ 1000 Ljubljana \\ Slovenija}
 
\KOMAoptions{fromphone=false,fromfax=false}
\setkomavar{subject}{Prošnja za dodelitev nepovratne denarne pomoči študentom v stiski.}

\opening{Spoštovani,}
obiskujem prvi letnik interdisciplinarnega magistrskega programa druge stopnje \textit{Računalništvo in matematika}, ki ga izvaja Fakulteta za matematiko in fiziko, skupaj s Fakulteto za računalništvo in informatiko v Ljubljani. Letos septembra sem diplomiral na dodiplomskem programu. Študij mi je zelo všeč in v prihodnosti se vidim kot zelo uspešnega na tem področju, zato sem se tudi odločil vpisati na magistrski program. Na Vas se obračam z upanjem, da mi boste lahko pomagali pri reševanju hude socialne stiske, ki je v tem trenutku nevzdržna. S tem bi mi veliko prispevali k vzpostavitvi dostojne kakovosti življenja.
\\
\\
V skupnem gospodinjstvu živim z materjo, ki je že dolgo ločena. Mater pestijo številne zdravstvene težave, ki izhajajo iz akutnega obolenja hrbtenice, išjasa in psihičnih težav. Zaradi tega je invalidsko upokojena za štiri urni delovni čas, za kar prejema invalidsko pokojnino v višino cca. \EUR{230} mesečno. Druge štiri ure je zaposlena v Adrii Mobil d.o.o. na prilagojenem delovnem mestu, za kar prejema cca. \EUR{300} mesečno. V kolikor pa se nahaja v bolniškem staležu, kar je zaradi hudih zdravstvenih težav pogosto, pa je plača še nižja.
\\
\\
Oče se nahaja v socialno varstvenem zavodu Hrastovec - bivalna enota Gornja Radgona, saj kjub dvoletni hospitalizaciji v psihiatrični kliniki v Ljubljani ni zmožen skrbeti sam zase. Zaradi bolezni je invalidsko upokojen. Po očetu prejemam zakonsko določeno preživnino v višini \EUR{197}, preostanek njegove pokojnine pa niti ne zadostuje za celotno pokritje stroškov bivanja v zavodu, tako da od njega ne morem pričakovati nobenih dodatnih sredstev.

\pagebreak

Skupni mesečni dohodki, ki ju prejemava z mamo, znašajo okoli \EUR{727} mesečno. Od tega znaša najemnina za stanovanje v Novem mestu \EUR{177}, najemnina za študentsko sobo \EUR{60} in ostali osnovni stroški (elektrika, ogrevanje, kuhinjski plin, komunala, voda, komunikacije) okoli \EUR{200}. 
Po poplačilu vseh stroškov nama za preživetje skozi mesec ostane okoli \EUR{290}, kar ni praktično nič.
\\
\\
Od oktobra letos pa je socialna stiska postala še hujša, saj moram za potrebe študija bivati v Ljubljani, kar nanese dodatne stroške. Za vikend hodim domov, saj se mame zaradi njenih zdravstvenih težav bojim pustiti predolgo same. Cena železniške vozovnice na relaciji Ljubljana - Novo mesto znaša \EUR{6,22}. Cena vozovnice Ljubljanskega mestnega potniškega prometa, da lahko pridem od kraja bivanja do železniške postaje znaša \EUR{1,20}. Skupaj je to \EUR{7,42} v eno smer oz. \EUR{14,84} v obe smeri. Vikendi so najmanj štirje v mesecu, zato stroški prevoza skupaj znesejo najmanj \EUR{59,36} (cca. \EUR{60}). Na dan si komaj privoščim topel obrok preko študentskih bonov v študentski menzi v Rožni dolini. Cena bona je \EUR{2,4}. Na mesec imam na voljo okoli 20 bonov (brez vikendov), zato strošek znaša cca. \EUR{50}. Tu seveda ni vštet nakup ostalih življenskih potrebščin, kar seveda stroške še zviša. Doma ne pridelujemo nobene hrane niti za to nimamo možnosti, saj z mamo živiva v bloku, zato si morava vso hrano kupovati. Poleg tega pa za potrebe študija nujno potrebujem prenosni računalnik, ki se mi zaradi obremenjenosti neprestano pregreva in kvari in je delo na njem skoraj nemogoče. Za popravilo in očiščenje potrebujem okoli \EUR{80}, ki jih nimam od nikjer vzeti. 
\\
\\
Mati zaradi svojih hudih zdravstvenih težav ne more sprejeti nobenih dodatnih del, ki bi lahko zadostile finančno stisko. Zaradi intenzivnega študija in želje, da v tekočih rokih opravim vse svoje študijske obveznosti si tudi jaz ne morem privoščiti občasnih del. Pri tem me še dodatno omejuje dejstvo, da moram med vikendom, ko sem doma, skrbeti za svojo mater.
\\
\\
Letos novembra sem prejel pozitivno odločbo za državno štipendijo centra za socialno delo Novo mesto, v višini \EUR{215,84}. V študijskem letu 2012/2013 štipendije nisem prejemal, prav tako pa prvo nakazilo štipendije pričakujem šele decembra, zato je pri opisu socialne stiske nisem upošteval.
\\
\\
Trenutno me najbolj boli dejstvo, da si sposojam denar za plačilo študentske sobe, za prevoz iz Novega mesta v Ljubljano in nazaj in tudi za osnovne življenjske potrebščine. S prvim nakazilom štipendije bom deloma povrnil izposojen denar. Ob vsem tem si tudi ne morem privoščiti obiska svojega očeta, saj nimam denarja za pot, kar me psihično zelo utruja. Prav tako se vse manj družim z vrstniki, saj se zaradi svoje stiske počutim odrinjenega.

\pagebreak

Na podlagi zgoraj navedenega Vas vljudno prosim za dodelitev nepovratne denarne pomoči študentom v stiski v višini \EUR{300}, za kar Vam bom izredno hvaležen. Denarna pomoč bi mi zelo pomagala pri poplačilu vseh obveznosti in me s tem rešila finančnih bremen, ki negativno vplivajo na moj študij in kvaliteto življenja. Omogočila bi mi tudi, da obiščem očeta v socialno varstvenem zavodu Hrastovec, ki ga zaradi finančne stiske nisem obiskal že več kot pol leta.
\\
\\
Denarno pomoč nameravam porabiti na naslednji način:
\begin{itemize}
\item \EUR{60} - Plačilo najemnine za študenstko sobo;

\item \EUR{30} - Avtobusna karta Ljubljana-Gornja Radgona in nazaj za obisk očeta v socialno varstvenem zavodu Hrastovec - bivalna enota Gornja Radgona. \\
\textit{Vir:} \url{http://www.ap-ljubljana.si/};

\item \EUR{60} - Mesečni strošek prevoza med Ljubljano in Novim mestom z vlakom in strošek uporabe Ljubljanskega mestnega potniškega prometa do železniške postaje in nazaj, da lahko za vikend hodim domov in skrbim za svojo mater, ki ima hude zdravstvene težave. \\
\textit{Vir:} \url{http://www.slo-zeleznice.si/} in \url{http://www.lpp.si/sites/default/files/lpp_si/stran/datoteke/cenik_vozovnic_v_mestnem_in_integriranem_potniskem_prometu.pdf};

\item \EUR{80} - Strošek za popravilo in očiščenje prenosnega računalnika, ki se zaradi pregrevanja neprestano kvari;

\item \EUR{70} - Nakup osnovnih življenskih potrebščin in prehranjevanje preko študentskih bonov v študentski menzi v Rožni dolini v Ljubljani. \\ \textit{Vir:} \url{http://www.studentska-prehrana.si} 
\end{itemize}



\closing{S spoštovanjem,}

\newpage
\textbf{Priloge:} \\
\begin{enumerate}
	\item Izpolnjena vloga za dodelitev nepovratne denarne pomoči;
	\item Izpolnjena razpredelnica o namenu porabe sredstev;
	\item Fotokopija osebnega dokumenta z obeh strani;
	\item Podpisana izjava o nezaposlenosti;
	\item Potrdilo o vpisu na Fakulteti za matematiko in fiziko v Ljubljani;
	\item Fotokopija bančnih kartic z obeh strani;
	\item Podpisana izjava o številu osebnih računov;
	\item Dokazilo banke o vseh nakazilih in prejemkih od 01.05.2013 do 31.10.2013 - NLB;
	\item Dokazilo banke o vseh nakazilih in prejemkih od 01.05.2013 do 31.10.2013 - Abanka;
	\item Izpis zaslužkov, prejetih na podlagi študentske napotnice od 01.05.2013 do 31.10.2013 in izjava o aktivnem članstvu v študentskem servisu - E-študenstki servis;
	\item Potrdilo Upravne enote Novo mesto o skupnem gospodinjstvu
	\item Obrazložitev potrdila o skupnem gospodinjstvu;
	\item Fotokopija dohodninske odločbe za leto 2012 - Dara Klepić;
	\item Fotokopija potrdila Davčne uprave Novo mesto o prejetih dohodkih za leto 2012 - Igor Klepić;
	\item Fotokopija računa za najemnino stanovanja v Novem mestu;
	\item Fotokopija računa za najemnino študentske sobe v Ljubljani;
	\item Mnenje Centra za socialno delo Novo mesto;
	\item Fotokopije zdravniških izvidov matere Dare Klepić.
	
	
\end{enumerate}

\vspace{5mm}





 
\end{letter}
 
\end{document}