\documentclass[a4paper]{scrlttr2}
\usepackage{lmodern}
\usepackage[utf8]{inputenc}
\usepackage[T1]{fontenc}
\usepackage[slovene]{babel}
\usepackage{url}
\usepackage{eurosym}

\enlargethispage{3\baselineskip}
\voffset = 20pt
\setlength{\parindent}{0cm}

 
 
\setkomavar{fromname}{Igor Klepić}
\setkomavar{fromaddress}{Ulica Slavka Gruma 92 \\ 8000 Novo mesto \\ Slovenija}
\setkomavar{fromphone}{0123 45679}
\setkomavar{fromemail}{igorklepic@gmail.com}

 
\begin{document}
 
\begin{letter}{Fakulteta za matematiko in fiziko\\ prof. dr. Sergio Cabello \\ Jadranska ulica 19 \\ 1000 Ljubljana \\ Slovenija}
 
\KOMAoptions{fromphone=false,fromfax=false,fromemail=true}

\setkomavar{subject}{Motivacijsko pismo za Erasmus izmenjavo 2014/15 - Praga.}

\opening{Spoštovani prof. dr. Sergio Cabello,}
obiskujem prvi letnik interdisciplinarnega magistrskega programa druge stopnje \textit{Računalništvo in matematika}. Lani septembra sem diplomiral na dodiplomskem programu. Študij mi je zelo všeč in v prihodnosti se vidim kot zelo uspešnega na tem področju, zato sem se tudi odločil vpisati na magistrski program.
\\
\\
Poleg študija sem dejaven še na veliko ostalih področjih. Že od majhnega igram šah, kjer sem bil uspešen na številnih državnih in mednarodnih tekmovanjih. Zelo rad se ukvarjam s športom - najbolj priljubljene so mi borilne veščine, pred kratkim pa sem  se tudi začel seznanjati s finančnimi trgi in učinkovitim upravljanjem premoženja. Na to temo sem tudi napisal diplomsko nalogo. Na področju računalništva letos sodelujem s podjetjem ComTrade - na njihovo pobudo sem v okviru predmeta na fakulteti raziskal težave, ki jih imajo pri algoritmu za strojno učenje, delo na projektu izven okvirjev fakultete pa še poteka.
\\
\\
Zelo rad potujem, saj z veseljem spoznavam nove kulture in jezike. Aktivno govorim angleško, nemško in srbsko, pasivno pa tudi špansko in makedonsko. Na Češkem še nisem bil in eden izmed razlogov je tudi ta, da bi se rad seznanil s Češko kulturo in jezikom. Poleg vsega pa si zelo želim neko daljše obdobje preživeti v tujini in čas študija je kot nalašč primeren za tako izkušnjo.

\pagebreak

Tekom študija v Ljubljani sem srečal veliko študentov, ki študirajo v \textit{Charles University of Prague} in so prišli na izmenjavo v Ljubljano. Navdihnili so me, da si za izmenjavo izberem to univerzo, saj ima dolgoletno tradicijo na področju matematike, prav tako pa ima tudi zelo kvalitetne in raznolike programe na področju računalništva, kar je kjučnega pomena za mojo smer študija. Poleg tega pa je Praga ena izmed pomembnejših zgodovinskih, kulturnih in izobraževalnih Evropskih središč, ki nudi obilo priložnosti za nas mlade.
\\
\\
Študentsko izmenjavo vidim kot priložnost za ustvarjanje novih poznanstev, spoznavanje drugačnega načina življenja, učenja tujega jezika in kot pomembno življensko izkušnjo, ki bi mi močno izboljšala kvaliteto študija in posledično tudi zaposlitvene možnosti v prihodnosti. Danes je zelo zaželjeno imeti izkušnje iz tujine, kmalu pa bo to skoraj da nujno, zato mi je tovrstna priložnost še toliko bolj pomembna.
\\
\\
Z izmenjavo bi mi omogočili, da izkusim življenje v tujini, ki si ga zelo želim. Pridobil bi kompetence na področjih, ki jih drugače ne bi mogel osvojiti. Razširil bi si obzorja na svojem področju študija kot tudi na vseh ostalih področjih življenja in tako bi svojo študijsko in karierno pot nadaljeval z boljšim znanjem in bogatejšimi izkušnjami. 





\closing{S spoštovanjem,}

\newpage





 
\end{letter}
 
\end{document}